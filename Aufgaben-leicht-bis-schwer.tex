\documentclass{article}
\usepackage[utf8]{inputenc}
\usepackage[german]{babel}
\usepackage{xcolor}
\usepackage{graphicx} % Required for inserting images
% \usepackage{tikz}
\usepackage{multicol}
\usepackage{setspace}

\title{Gemischte Matheaufgaben }

\newcommand{\addition}[2]{
$ {#1} + {#2} = $

}

\newcommand{\subtraktion}[2]{
$ {#1} - {#2} = $

}


\begin{document}
\section{Unter 10 rechnen}

\doublespacing
\begin{multicols}{3}


\addition{1}{5}

\addition{3}{5}

\subtraktion{4}{2}

\addition{2}{4}

\subtraktion{9}{3}

$2 + 3 - 5 = $

\end{multicols}

\singlespacing
\section{Über 10 rechnen}

\doublespacing
\begin{multicols}{3}
$ 5 + 9 = $

$ 16 - 8 = $

$ 3 + 6 + 4 = $

$ 8 + 7 = $

$ 13 - 9 = $

$ 17 - 2 - 6 = $

\end{multicols}

\singlespacing

\section{Bis 100 rechnen}

\doublespacing

\begin{multicols}{3}
$14 + 15 = $

$39 + 21 = $

$90 - 20 = $ 

$77 - 53 = $

$35 - 17 = $

$99 - 33 + 26 = $
\end{multicols}

\singlespacing

\section{Kleines Einmaleins}

\doublespacing

\begin{multicols}{3}
$2 \cdot 3 = $

$5 \cdot 5 = $

$7 \cdot 7 = $

$3 \cdot 4 = $

$4 \cdot 9 = $

$6 \cdot 8 = $

$8 \cdot 4 = $

$9 \cdot 5 = $

$1 \cdot 8 = $
\end{multicols}
\singlespacing

\section{Division (Einmaleins rückwärts)}

\doublespacing

\begin{multicols}{3}
$18 : 2 = $

$64 : 8 = $

$36 : 4 = $

$21 : 7 = $

$72 : 9 = $

$49 : 7 = $

$63 : 9 = $

$15 : 3 = $

$50 : 5 = $
\end{multicols}
\singlespacing


\end{document}
